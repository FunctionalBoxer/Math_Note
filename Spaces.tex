\documentclass[8pt]{report}
\usepackage[english]{babel}

\usepackage[letterpaper,top=1.2cm,bottom=1.2cm,left=1.2cm,right=1.2cm,marginparwidth=1.75cm]{geometry}

% Useful packages
\usepackage{amsmath, amsthm, amssymb, graphicx, enumitem, authblk, tikz, tikz-cd, verbatim, relsize, }
\usepackage{tcolorbox}
\usepackage{tikz-cd}
\usepackage{forest} % 트리 구조를 위한 패키지
\usepackage{mathtools}
\usepackage[colorlinks=true, allcolors=black]{hyperref}
\usepackage[framemethod=TikZ]{mdframed}
\usepackage{lmodern}
\DeclareMathAlphabet{\mathtt}{OT1}{lmtt}{m}{n}

\usepackage{fontspec}
\setmainfont{neodgm_code.ttf}[
  Path = ./,
  UprightFont = *,
  Ligatures = TeX,
  Scale = MatchLowercase,
  % 폰트에 Bold/Italic이 없을 때 임시로:
  AutoFakeBold = 2.0,
  AutoFakeSlant = 0.2
]

\usetikzlibrary{calc}
\usetikzlibrary{shapes,backgrounds}
\usetikzlibrary{decorations.pathreplacing}
\usetikzlibrary{arrows.meta, positioning, matrix, fit}

% \usetikzlibrary{arrows.meta, positioning, external}
% \pgfplotsset{compat=1.18} % Set compatibility for pgfplots

\newtheoremstyle{romanstyle} % 스타일 이름
  {10pt} % 위쪽 여백
  {10pt} % 아래쪽 여백
  {\rmfamily} % 본문 로마체
  {} % 들여쓰기 없음
  {\bfseries} % 제목 굵게
  {.} % 제목 뒤 구두점
  { } % 제목과 본문 사이의 기본 간격
  {} % 제목 레이아웃 기본값

% 새 스타일 적용
\theoremstyle{romanstyle}
\newtheorem{proposition}{Proposition}
\newtheorem{theorem}{Theorem}
\newtheorem{lemma}[theorem]{Lemma}
\newtheorem{ntlemma}{Lemma(Number Theory)}
\newtheorem{definition}{Definition}
\newtheorem{corollary}[theorem]{Corollary}
\newtheorem{axiom}{Axiom}

% Unnumbered environments
\newtheorem*{observation}{Observation}
\newtheorem*{example}{Example}
\newtheorem*{solution}{Solution}

% Mathematical commands
\newcommand{\st}{\text{ s.t. }}
\newcommand{\as}{\text{ as }}
\newcommand{\setx}{\{x_0, x_1, \dots, x_n\}}
\newcommand{\setn}{\{1, \dots, n\}}
\newcommand{\dis}{\displaystyle}
\newcommand{\lete}{Let $\epsilon > 0$ be given.}
\newcommand{\fs}{\text{ for some }}
\newcommand{\defequal}{\overset{\textnormal{def}}{=}}
\newcommand{\setequal}{\overset{\textnormal{set}}{=}}
\newcommand*{\aand}{_{\text{ and }}}
\newcommand*{\oor}{_{\text{ or }}}
\DeclareMathOperator{\argmax}{argmax}
\DeclareMathOperator{\argmin}{argmin}
\newcommand{\im}{\operatorname{im}}
\newcommand{\nullity}{\operatorname{nullity}}
\newcommand{\rank}{\operatorname{rank}}
\newcommand{\ipvs}{\langle \, \cdot \, \rangle}
\newcommand{\diam}{\operatorname{diam}} % Add this line in the preamble

\title{Math Note}
\author{Jong Won}
\affil{University of Seoul, Mathematics}
\date{}

\begin{document}
\maketitle

\newpage
\tableofcontents % Generate table of contents
\vspace{3cm}

This paper contains independent topics in undergraduate mathematices.

\chapter{Set Theory}
\chapter{Group Theory}
\begin{example}
    \textbf{Dihedral Group}
\end{example}
\chapter{Ring Theory}
\newpage
\section{Ring of Fractions}
\begin{theorem}
    Let $R$ be a Commutative Ring, $D \subset R$ be a subset such that $\begin{cases}
        \text{no zero, no zero divisors} \\ \text{closed under multiplication}
    \end{cases}$. \\
    Then, there exists a Commutative Ring $Q$ with identity satisfies:
    \begin{enumerate}
        \item $R$ can embed in $Q$, and every element of $D$ becomes unit in $Q$.
        More precisely, $Q = \{r d^{-1} \mid r \in R,\ d \in D\}$.
        \item $Q$ is the smallest Ring with identity such that every element of $D$ becomes unit in $Q$.
    \end{enumerate}
\end{theorem}
\begin{proof}
    Let $\mathcal{F} \defequal \{(r,d) \mid r \in R,\ d \in D\}$ and the relation $\sim$ on $\mathcal{F}$
    by $(r_1, d_1) \sim (r_2, d_2) \iff r_1d_2 = r_2d_1$.\\
    Then, $\sim$ is equivalent relation: reflexive and symmetirc are clear, and
    Suppose that $(r_1, d_1) \sim (r_2, d_2)$ and $(r_2, d_2) \sim (r_3, d_3)$.
    \begin{align*}
        r_2d_3 = r_3d_2 \implies r_2 d_1 d_3 = r_3 d_1 d_2 \implies r_1 d_2 d_3 = r_3 d_1 d_2 \implies d_2 ( r_1 d_3 - r_3 d_1) \implies r_1d_3 = r_3d_1
    \end{align*}
    Thus transitivity shown. Define
    \begin{align*}
        \frac{r}{d} \defequal [(r,d)] = \{(a,b) \mid (a,b) \sim (r,d) \},\ Q \defequal \left\{ \frac{r}{d} \;\middle|\; r \in R,\ d \in D \right\}
    \end{align*}
    And define operations $+, \times$ on $Q$:
    \begin{align*}
        \frac{r_1}{d_1} + \frac{r_2}{d_2} \defequal \frac{r_1d_2 + r_2d_1}{d_1 d_2}, \quad
        \frac{r_1}{d_1} \times \frac{r_2}{d_2} \defequal \frac{r_1r_2}{d_1d_2}
    \end{align*}
    Well-Definedness: If $\dis \frac{r_1}{d_1} = \frac{r'_1}{d'_1}$ and $\dis \frac{r_2}{d_2} = \frac{r'_2}{d'_2}$,
    \begin{align*}
        \frac{r_1d_2 + r_2d_1}{d_1 d_2}
        = \frac{r_1 d_2 d'_1 d'_2 + r_2 d_1 d'_1 d'_2}{d_1 d_2 d'_1 d'_2}
        = \frac{(r_1 d'_1) d_2 d'_2 + (r_2 d'_2) d_1 d'_1 }{d_1 d_2 d'_1 d'_2}
        = \frac{(r'_1 d_1) d_2 d'_2 + (r'_2 d_2) d_1 d'_1 }{d_1 d_2 d'_1 d'_2}
        = \frac{(r'_1d'_2 + r'_2d'_1)d_1d_2}{d_1d_2 d'_1 d'_2}
        = \frac{r'_1d'_2 + r'_2d'_1}{d'_1 d'_2}
    \end{align*}
    \begin{align*}
        \frac{r_1r_2}{d_1d_2} 
        = \frac{r_1 r_2 d'_1 d'_2}{d_1d_2d'_1d'_2}
        = \frac{(r_1 d'_1) (r_2 d'_2)}{d_1d_2d'_1d'_2}
        = \frac{(r'_1 d_1) (r'_2 d_2)}{d_1d_2d'_1d'_2}
        = \frac{r'_1 r'_2 d_1 d_2}{d_1d_2d'_1d'_2}
        = \frac{r'_1 r'_2}{d'_1d'_2}
    \end{align*}
    Now, $(Q, +, \times)$ constructs Commutative Ring with identity: for any $d \in D$, 
    put $\dis 0_Q \defequal \frac{0}{d},\ 1_Q \defequal \frac{d}{d}$. Then,
    \begin{enumerate}
        \item $(R, +, \times)$ closed under the operations since $D$ is closed under the multiplication.
        \item $(R, +)$ has a zero: $\dis \frac{r_1}{d_1} + 0_Q = \frac{r_1}{d_1} + \frac{0}{d} = \frac{r_1 d + 0 d_1}{d_1 d} = \frac{r_1 d}{d_1 d} = \frac{r_1}{d_1}$.
        \item $(R, +)$ has an inverse: $\dis \frac{r_1}{d_1} + \frac{-r_1}{d_1} = \frac{r_1d_1 + (-r_1)d_1}{d_1 d_1} = \frac{[(r_1) + (-r_1)]d_1}{d_1 d_1} = \frac{0 d_1}{d_1 d_1} = \frac{0}{d_1 d_1} = 0_Q$.
        \item $(R, +, \times)$ satisfies distributive law:
        \begin{enumerate}[label=\theenumi-\arabic*.]
            \item The left law:
            \begin{align*}
                \frac{r_1}{d_1} \times \left( \frac{r_2}{d_2} + \frac{r_3}{d_3} \right)
            = &\frac{r_1}{d_1} \times \frac{r_2 d_3 + r_3 d_2}{d_2 d_3}
            = \frac{r_1 r_2 d_3 + r_1 r_3 d_2}{d_1 d_2 d_3}
            = \frac{r_1 r_2 d_1 d_3 + r_1 r_3 d_1 d_2}{d_1 d_2 d_1 d_3}
            = \frac{r_1 r_2}{d_1 d_2} + \frac{r_2 r_3}{d_2 d_3} \\
            = &\frac{r_1}{d_1} \times \frac{r_2}{d_2} + \frac{r_2}{d_2} \times \frac{r_3}{d_3}
            \end{align*}
            \item The right law:
            \begin{align*}
                \left(\frac{r_1}{d_1} + \frac{r_2}{d_2}\right) \times \frac{r_3}{d_3}
            = &\frac{r_1 d_2 + r_2 d_1}{d_1 d_2} \times \frac{r_3}{d_3}
            = \frac{r_1 r_3 d_2 + r_2 r_3 d_1}{d_1 d_2 d_3}
            = \frac{r_1 r_3 d_2 d_3 + r_2 r_3 d_1 d_3}{d_1 d_3 d_2 d_3}
            = \frac{r_1 r_3}{d_1 d_3} + \frac{r_2 r_3}{d_2 d_3} \\
            = &\frac{r_1}{d_1} \times \frac{r_3}{d_3} + \frac{r_2}{d_2} \times \frac{r_3}{d_3}
            \end{align*}
        \end{enumerate}
        \item $(R, \times)$ has an identity: $\dis \frac{r_1}{d_1} \times 1_Q = \frac{r_1}{d_1} \times \frac{d}{d} = \frac{r_1d}{d_1d} = \frac{r_1}{d_1}$.
        \item Elements of $D$ become unit in $Q$: Define $\iota:R \to Q:r \mapsto \dis \frac{rd}{d}$ where $d \in D$ is any fixed element in $D$. \\
        Then, $\iota$ is Ring-Monomorphsim because:
        \begin{enumerate}[label=\theenumi-\arabic*.]
            \item Well-Defined and Injective: 
            $\dis \iota (r_1) = \iota (r_2)
            \iff \frac{r_1 d}{d} = \frac{r_2 d}{d} 
            \iff (r_1 - r_2) d d= 0
            \iff r_1 = r_2$
        \end{enumerate}
    \end{enumerate}    
\end{proof}
\chapter{Field Theory}

\section{Check;}
Check to upload:
\chapter{Category}
\chapter{General Topology}
\chapter{Algebraic Topology}
\chapter{Real Analysis}
\chapter{Measure}
.
\chapter{Complex Analysis}
\chapter{Differential Geometry}
\chapter{Differential Equation}


\end{document}
